\section{METHODOLOGY}
\subsection{Simulations and Sample Selection}
As a simplest estimation for the resolution for the single dish telescope, we use the formula:
\begin{equation}
    \theta = \frac{\lambda_z}{D},
\end{equation}
in which $\theta$ is the angular resolution, $\lambda_z$ is the wavelength of the redshifted observed signal, and $D$ is the effective diameter of the radiotelescope. Using Hubble constant $ H_0 = 67.4 \text{ km s}^{-1} \text{ Mpc}^{-1}$, $\Omega_m = 0.3$, and $\Omega_\Lambda = 0.7$, we can calculate the corresponding angular resolution for our observations, which is shown in table \ref{tab:angular_resolution}.

\begin{table}[h]
    \caption{Redshifted 21-cm line frequencies, angular resolutions and comoving perpendicular resolutions of the single dish telescopes at different redshifts.}
    \label{tab:angular_resolution}
    \centering
    \begin{tabular}{cccccccccc}
        \toprule
        \multirow{2}{*}{$D/\text{m}$}&\multicolumn{3}{c}{$z=5$}&\multicolumn{3}{c}{$z=8$}&\multicolumn{3}{c}{$z=12$}\\
        \cmidrule(lr){2-10}
        &$f/\text{MHz}$&$\theta$&$d/\text{Mpc}$&$f/\text{MHz}$&$\theta$&$d/\text{Mpc}$&$f/\text{MHz}$&$\theta$&$d/\text{Mpc}$\\
        \midrule
        300         &\multirow{2}{*}{236}&$0.242^\circ$&$34.1$&\multirow{2}{*}{158}&$0.363^\circ$& $58.9$&\multirow{2}{*}{109}&$0.524^\circ$&93.3\\
        100         &&$0.726^\circ$&$102$ &&$1.08^\circ$& $177$&&$1.57^\circ$&280\\
        \bottomrule
    \end{tabular}
\end{table}

The typical bubble size during the EoR is around $O(10 \text{Mpc})$. From table \ref{tab:angular_resolution}, we can see that the single dish telescope with diameter $D=100 \text{m}$ is only capable to resolve the structure of the size of $~O(100 \text{Mpc})$, which is much larger than the typical bubble size. Therefore, in this work, we only focus on the average spectrum along different lines of sight, instead of the fluctuations perpendicular to the line of sight. It might be a concern that $O(100 \text{Mpc})$ has reached the cosmological scale, and it will make the signal fluctuations relatively small. Luckily, the receiver of the radio telescope could have a very high frequency resolution, which means that we are actually doing average across a very thin slice perpendicular to the line of sight for a certain frequency bin or redshift bin. The comoving distance resolution along the line of sight direction is given by: